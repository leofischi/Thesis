\chapter{General EMG-driven controller characteristics}

The main focus of this present work is to design a control method that allows the user to control an exoskeleton. 

The ideal control method would be one that requires no prior training. The user would be capable of controlling the mechanism as easily as he can control his own limb. Of course this is an ideal scenario and, as stated in many previous studies already cited in this work, we are still far from understanding the real dynamics of limbs, muscles and electromyography signals.

With those challenges in mind, how can we design a control that is capable of controlling a mechanical "limb-like" mechanism?

The first idea that comes to mind is to design a control method that mimics the physical characteristics of the human limb, that is, a biomimetic control. Biomimetics is the study of biological mechanisms and processes with the purpose of synthesizing similar products and behaviors by an artificial mechanism which mimics natural ones \cite{merriamWebster}. To achieve this biomimetic behavior, the model-based control method is proposed.

Many times for the control of machines, like cars, the control systems do not use biomimetic mechanisms. Instead, an easier mechanism or system is designed so that, with the proper training, the user is capable of controlling the machine even in complex activities. With that idea in focus, we can use the proportional EMG control.

As many prior studies have stated, the usage of hybrid methods can enhance the accuracy and ease-of-use of the controller. In this work, a control method using sEMG and force sensors will be applied.

For this work, these three control methods will be designed, applied to the mechanism and evaluated.












